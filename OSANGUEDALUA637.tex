\chapterspecial{O {sangue} {da} {lua}}{}{}
 

\letra{N}{o início,} os dois que flecharam a Lua também existiam, antes de nossos
antepassados Yanomami se misturarem. Eram espíritos. O irmão mais novo,
Uhutimari, morava com seus irmãos. Eram somente eles, junto ao seu irmão
mais velho, Escorpião. Eram três. O mais velho tinha o nome daquele
inseto que faz doer muito, o escorpião. Por isso, chamava"-se Escorpião.
O do meio tinha o nome da árvore paricá.\footnote{ A árvore paricá fornece as sementes e a casca com as quais os Yanomami
produzem um pó alucinógeno utilizado em diversos rituais.} 

Quem realmente flechou a Lua, o verdadeiro flecheiro da Lua, foi
Escorpião. 

Por que a flechou?Naquela época, a Lua era baixae ficava sentada na
terra. Sendo muito faminta de carne, ela devorava sempre as crianças.
Devorou o filho de Paricá. 

Era alta, assim como hoje? Não! Perto da casa de Escorpião, erguia"-se um
jatobá reto, onde a Lua se empoleirou desajeitadamente. Ela se sentou em
uma forquilha baixa. Paricá e seu grupo moravam junto com os dois
outros, Escorpião e Uhutimari. A Lua devorava as filhas quase formadas,
os filhos quase crescidos, as filhas quase moças. Ela comia as crianças
dessa faixa etária. 

Como ela as queimava, chamando assim o ódio de todos, os dois irmãos
mais novos quase a flecharam quando ela desceu para
atacar{.} Paricá se deslocou de \emph{wayumɨ} e ensinou aos
demais a se deslocarem de \emph{wayumɨ.} É por isso que hoje os Yanomami
vão de \emph{wayumɨ} onde há o bacabal para se alimentar. Os irmãos iam
de \emph{wayumɨ} e, assim, nos ensinaram. Foram por lá.

O xapono deles era como o nosso. Ele chorou como nós choramos, ficou
muito abalado. Depois da cremação do corpo do segundo filho, que a Lua
comeu em seguida, Paricá cobriu as cinzas no meio do xapono. Saíram
de \emph{wayumɨ}. 

Em determinado momento, um dos integrantes teve de voltar correndo,
tendo esquecido os dentes de cutia, outros ficaram sentados a certa
distância. Quando chegou à entrada do xapono, ele viu a Lua comendo as
cinzas no meio do xapono, que a gente sempre mantém limpo. A massa da
Lua se mexia.

--- \emph{Ũũũũũ}--- fazia um ronco assim. 

Ela comia até o carvão, devorava gulosamente as cinzas. 

\emph{Hɨ}! Ele ficou com medo e retornou rapidamente: 

--- Será que o monstro grande está fazendo isso? Ele está comendo? O
monstro está comendo as cinzas! \emph{Sãrai}!--- disse retornando de
medo. Ele foi buscar e avisar o pai da criança morta, Escorpião, que
fará a Lua sofrer as conseqüências. Ele buscou o pai. 

--- O monstro grande está comendo lá! Ele está comendo o que fez você de
luto. Ele está comendo gulosamente os restos, ele está comendo as cinzas
do seu filho--- disse. 

--- Ele está comendo as cinzas do meu filho!?--- disse o pai, desolado e
chorando. 

--- Vamos! Vamos! Vamos, meu irmão!--- disseram os dois irmãos mais
novos, apesar de eles não serem bons flecheiros. 

--- Já aprontamos as pontas de nossas flechas--- disseram. 

Escorpião observava os dois flechando em vão, pensando que eles não
conseguiriam, apesar de a Lua não estar muito alta, pois os dois eram
péssimos flecheiros. 

A Lua se empoleirava e olhava para si mesma, porque tinha comido o
filho. A Lua estava mole, digerindo mal. 

\emph{Taɨ xiri, taɨ xiri, taɨ xiri}! Os dois estavam flechando, mas suas
flechas, infelizmente, não acertavam o alvo. Fizeram a Lua subir,
espantaram"-na. Fizeram"-na subir, de tantas flechas que atiraram. 

Ela ficou altíssima, rodando e subindo, e os dois insistindo. \emph{Tai,
tai, tai, tai}! A Lua fez as flechas se tornarem espíritos. Por fim,
Escorpião, o pai da criança morta, conseguiu vingá"-la. 

Nossos antepassados não sabiam fazer guerra, foi ele quem nos ensinou. 

A Lua subia em direção à sua casa, sua rede estava lá, lá em cima. A sua
casa e a sua região estavam escondidas. 

Quando a Lua, que era diferente desta, passou pela porta, ele a flechou.
Quando entrou, ela estava cansada e deitou"-se lentamente na sua rede. 

Escorpião se moveu, levantando"-se e ao mesmo tempo apontando a flecha
para cima. 

Quando ela estava pronta para entrar, quando ela ia se sentar na sua
rede: \emph{prãoo, kroxooo}! Ele não falhou: apesar da altura, ele
acertou plenamente. Apesar de o vento sempre soprar muito nessa região,
a flecha não desviou, a flecha voou direto através do vento. Quando a
Lua se inclinava para deitar, a flecha se fincou entre as duas
escápulas.Escorpião a fez balançar. O sangue jorrou. \emph{Ho, ho, ho,
ho, ho, ho, ooooooo}! A Lua! \emph{Taka, taka, taka, taka, taka,
taka}! \emph{Ha}! Ele a fez estremecer. 

O sangue caído, as gotas de sangue caindo de lá para cá não se
estragaram. O sangue caía se transformando logo em Yanomami, mas em
Yanomami ferozes. O sangue se transformou em Yanomami, que imediatamente
flechavam. As gotinhas de sangue voaram sem se espalhar bem. O sangue
desceu flechando e não se esgotou. Os Yanomami, transformados a partir
do sangue da lua, mataram os habitantes do xapono do flecheiro da Lua. 

Ninguém sobreviveu, nem Escorpião, que se tornou espírito. Os dois que
conviviam com Paricá tampouco sobreviveram. O Sangue da Lua queria se
tornar Yanomami; queria se tornar ferozes Yanomami. Queria se tornar
matadores de Yanomami. Aconteceu, assim é. 

Foi então que surgiram nossos antepassados, a partir do Sangue da Lua. 

Assim aconteceu: o Sangue da Lua exterminou todos os que moravam em
baixo. Somente os espíritos sobreviveram, apesar de eles viverem como
nós. Os Yanomami Sangue da Lua não guerrearam com os espíritos.

Pouparam Kasimi e seu neto. 

 
