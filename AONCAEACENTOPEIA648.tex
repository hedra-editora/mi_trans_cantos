\chapterspecial{A onça e a centopeia}{}{}
 

\letra{N}{essa época} as onças não comiam gente, não andavam, não existiam. Não
havia onça na floresta. Daí essa história. Não andava onça por aí para nos
matar e nos comer. Hu, Hu! Hu! A onça não dizia isso. 

Quem encontrou a primeira onça? Sozinha, ela sofria de fome, sequinha,
sua barriga gritava de fome, pois ela não tinha dente. Onça tinha apenas
gengivas, ela não mastigava, ela andava magra no meio dessa região do
Xererei, ela andava sozinha, andarilha, faminta. Como ela não comia quase
nada, ela chorava. Ela chorava por fome de carne. 

Quem a encontrou? Onça chegou onde estava Centopeia, onde morava
sozinha como gente; Onça chegou à casa de Centopeia. Ela apareceu, elas
se encontraram, ela ia de encontro. Com fome, andava como se fosse cego,
sem olhos, sofria mesmo, fazia muito barulho, tropeçava de fome. 

É uma centopeia! Vocês conhecem esse nome? Era gente, aquela que anda
sem fazer barulho. Krihi! Ninguém mais faz esse barulho, andando em cima de um pau.
Foi ela quem ensinou primeiro. 

Ela emprestou seus pés para Onça não fazer mais barulho; ela o ensinou a
andar discretamente. Depois do ensinamento de Centopeia, Onça andou, ela
foi lá, chegou à terra plana e desceu.

 
