\chapterspecial{{Monstro} {këyakëya}}{}{}
 

\letra{H}{avia também} os que viviam na região centro"-sul, os
Yãimoropɨwei, que ficaram presos, pois moravam dentro da terra com o
monstro Këyakëya--- que, portanto, não era gente. 

Os que asfixiaram Këyakëya existiam bem antes de nossos antepassados.
Këyakëya morava dentro da terra, na vizinhança do xapono dos
ancestrais.\footnote{ Os xaponos são as casas coletivas circulares onde moram os Yanomami. Cada casa corresponde a uma comunidade; em geral não se fazem duas casas numa mesma localidade.} 

Apesar de ser um monstro, Këyakëya era líder dos Yãimoropɨwei. Os
companheiros de Këyakëya moravam dentro da terra e a casa deles tinha um
respiradouro, como o da casa do tatu. A casa de Këyakëya também tinha um
respiradouro. Moravam ali também os Motuxi, que se dividiram e se
espalharam. 

Os Prãkiawëteri asfixiaram Këyakëya, tentando matá"-lo. Asfixiaram"-no,
foi assim que nos ensinaram a matar. Eles não o mataram com flecha. 

No início, não havia matança, não havia inimizade, não havia briga
mortal. Os \emph{napë} também não existiam.\footnote{  O termo \emph{napë} designa os estrangeiros, em geral os brancos, ou quem adotou seus costumes.} Os nossos antepassados não sabiam manifestar ira nem raiva. 

Ele conseguiu escapar sob a forma de espírito. Ele não se transformou à
toa. Os companheiros dele, como nós, sempre padeciam de fome; todos
morreram pela fumaça que entrou no buraco. 

Këyakëya nos legou o sentido de vingança por causa da filha de quem?
Qual é o nome do pai cuja filha foi vítima da crueldade de Këyakëya, que
chegou e entrou no xapono? A vítima que, brutalmente, Këyakëya fez
descer da rede e sair era a filha do líder dos Naɨyawëteri. Era uma moça
bonita, realmente muito bonita. Ela estava na primeira menstruação, e
mesmo assim, ele a arrancou da reclusão.

Apesar de ser monstro, Këyakëya existia e vivia como gente. Como
morava dentro de um buraco, depois de trucidar a menina menstruada, ele
e os demais membros do grupo foram asfixiados pelos Prãkiawëteri. Mas
apenas Këyakëya conseguiu fugir, se tornando eterno na forma de
espírito. Ele ainda existe como espírito. 

Naɨyawë desgalhava um pé de fruta \emph{naɨ}\footnote{ Segundo Lizot, uma balateira, \emph{Manilkara bidentata}.} em uma roça
distante. Aooo, aoooo, aoooo, aooo! Fazia assim para sua gente. 

Enquanto eles comiam a fruta \emph{naɨ}, Këyakëya arrancou a menina do
seu recluso, matou"-a e a devorou. Ele a comeu sozinho. 

Fez lascas pequenas da carne das demais crianças, que também havia
trucidado, para oferecer a todos seus companheiros. Amontoou as lascas de
carne que ele colocou no seu grande cesto, chamado \emph{yotema}.
Carregou todos os restos das crianças massacradas e levou junto o irmão
da menina menstruada, que estava vivo e bonito. Ele o fez sentar em cima
dos cadáveres dentro do cesto. 

O menino vivo, que ele levou, transformou"-se em papagaio durante o
percurso. Këyakëya saiu do xapono dos Naɨyawëteri e andava a passos
largos, foi então que o menino, já de longe, disse: 

--- Kuao! Kuao! Kuao!

Esse som se tornou o som dos papagaios. Esses pássaros voam; ele pousou
em um galho e assim ficou. Këyakëya olhou para a beira do cesto,
querendo ver se o menino ainda estava sentado. Fez o filho de Naɨyawëse
tornar papagaio. Como o menino não estava, ele retornou àquela direção.
O menino se tornou a imagem do papagaio que grita: Kuao! Kuao!
Kuao!

--- Ouça! Meu xerimbabo! Onde você pousou? Kuato, kuato, kuato! ---
disse Këyakëya voltando e correndo.--- Em qual paragem você
ficou? Kuato, kuato, kuato!

--- Õiyaoooo!--- disse o papagaio. 

Assim disse aquele que, apesar de ser filho de gente, tornou"-se
papagaio. 

É a história dos antepassados. Tambem existiam monstros com outros
xaponos, sendo essa a história de Këyakëya e dos Yãimoropɨwei, que
moravam em xaponos pouco distantes um do outro. 

Depois, aparecerá o nome do rio que tirará e levará muitos ancestrais
Yanomami. É somente depois da história dos Yanomami levados pelo rio que
vem nossa história. Os Waika a contam de uma maneira diferente, eles a
contam conforme seus antepassados lhes contaram.\footnote{   O par \emph{waika}/\emph{xamatari} parece ter sido usado originalmente para designar outros grupos yanomami vivendo em região geográfica diversa de quem fala, os primeiros ao norte e oeste, e os segundos ao sul, reconhecendo-se neles conjuntos de características que os particularizam. Os termos foram atribuídos em diferentes momentos pelos brancos para designar grupos específicos de forma estável e, no caso de \emph{xamatari}, para designar a própria língua do tronco yanomami usada pelos Parahiteri que fizeram este livro.} 

Os companheiros de Këyakëya não sobreviveram, morreram todos pela
fumaça. Eles os asfixiaram a todos, somente Këyakëya sobreviveu, se
transformando em espírito eterno. Esse sobrevivente alcançou o xapono dos
espíritos, pois se tornou um deles, quando ainda eram Yanomami e moravam
como nós. Ele os alcançou e ficou lá. 

Não mora mais onde o asfixiaram. Somente restou o marco dele. Não pensem
que os companheiros de Këyakëya sobreviveram e se agruparam enquanto ele
alcançava os espíritos! 

Não houve sobreviventes do grupo dos Naɨyawëteri. Acontecerá depois. Os
sobreviventes eram os que afundaram, não os outros antepassados. As
águas sobem devagar e os que afundam são os únicos sobreviventes. 

Depois, os que tinham o mesmo nome que as montanhas também
sobreviveram.