\chapterspecial{O {surgimento} {das} {cobras}}{}{}
 

\letra{N}{essa época também}, as cobras não rastejavam como rastejam hoje, elas
viviam como os Yanomami. Transformaram"-se onde desceu o Sangue da Lua,
na floresta. Lá, caíram as cobras que picam. Transformaram"-se em cobras
lá em cima, enquanto iam para uma festa. Hoje, quando vocês olham para o
céu, vocês veem o peito daqueles que se transformaram em cobras. Não
havia cobras, nem jiboias, nem sucurijus. Os poraquês não existiam, nem
os peixes. Nós comemos a carne de gente.

Eles se transformaram em cobra, não no xapono, mas nesta floresta mesmo.
Foram chamados e foram lá, Wataperariwë e Jibóia, o irmão mais velho.
Foram lá longe com as Cobras, mas se transformaram na floresta. Eles,
então, não foram dançar. 

Com a cabeça coberta de penas brancas, dessa mesma forma que nós nos
pintamos, cada um pintou seu corpo com listras diferentes. As Cobras
moravam na sua própria região, como gente. Transformaram"-se quando
foram convidadas a dançar. Elas antes viviam como gente. 

Quem eram os dois tuxauas? O irmão mais velho e o irmão mais novo
moravam com as Cobras. Os dois também foram dançar. Watawatariwë e
Jiboia moravam com seu grupo, as Cobras. Jiboia era o irmão do meio.
Watawatariwë era o caçula. Os dois irmãos mais velhos eram esses: Jiboia
e Sucuriju, que nasceu primeiro. Aqueles que se pintaram eram três, pois
havia também Watawatariwë, o caçula, por isso nós nos pintaremos assim. 

Os Parawari também viviam com eles. Por causa deles se metamorfosearam,
porque os Parawari os levaram. 

Todos eles moravam em frente à serra Wãyapoto, que ainda tem esse nome.
Ocupavam essa região ao pé da serra na planície. Eram todos bonitos. É o
nome da região onde moravam os antepassados. É o verdadeiro nome dessa
região. As Cobras bebiam a água do rio Wãyapo, tomavam banho, se lavavam
nesse rio bonito. Tomavam banho e bebiam água. 

Eles nos ensinaram, assim, a dançar, mas, infelizmente, se
metamorfosearam. Eles iriam dançar, mas, infelizmente, se transformaram.
Iriam dançar. Transformaram"-se em cobras imediatamente. Tornaram"-se
cobras. Não foram dançar no xapono de outros. 

Em que xapono iam dançar? No xapono daqueles que se transformaram, que ainda existe na terra plana. Aqueles que se transformaram, apesar de se
pintarem fora do xapono, sofreram a metamorfose, transformaram"-se em
cobras. 

Os que convidaram as Cobras, como se chamavam esses antepassados? Eles
gritavam enquanto cozinhavam o mingau de banana para os visitantes. 

--- Por que estão agindo assim?--- Perguntaram"-se.

Pareciam gritar de propósito. Transformaram"-se perto do xapono dos
Jalouaca. Transformaram"-se perto desse xapono. Transformaram"-se. Os
antepassados se chamavam assim, Jalouaca. Assim se chamava o líder. Eram
espíritos, são nomes de espírito. Eram Yanomami e moravam como os
Yanomami. 

Apesar de morarem, assim, como nós, após a metamorfose em cobra eles não
voltaram à condição de seres comuns. Pintaram"-se fora do xapono dos
Jalouaca, pensando: 

--- Os Yanomami se pintarão assim! 

E se pintaram com listras. Pintaram"-se, na parte superior do braço, com
cor de sangue preto, igual à cor de meu irmão mais novo, como a cor de
seu braço. As cobras \emph{maraxari} se pintaram assim; a cobra coral
também se pintou com manchas vermelhas. 

O segundo grupo do xapono das Cobras se pintou em outro lugar, distante,
para que aquelas do outro grupo, que se achavam bonitas, se zangassem.
Elas tomaram banho no rio Wataperari, cuja água era branca. Ficaram onde
brilhava a luz. Assim era a luz do rio. Perto do xapono dos Jalouaca,
havia o rio, o rio apareceu de repente. 

Um pouco longe do xapono, as outras Cobras se pintavam juntas. 

Pintaram"-se. No segundo grupo havia uma mulher. Os bonitos desse
grupo, eram muito bonitos, chegaram até as outras cobras. Chegaram
também com eles dois Parawari bonitos, todos eram muito bonitos.
Chegaram. A beleza de suas pinturas incomodou os outros, que ficaram com
inveja. Chegaram, enquanto os outros se pintavam com riscas. Aquele,
cujo nome eu dei, apareceu no meio deles, Sucuriju. Ele, o irmão mais
velho, estava ao final dos que chegavam, aquele que tem grandes
desenhos.

--- Hɨ̃hɨ̃! Wĩsa! Wĩsa!--- assobiaram. 

Os do primeiro grupo, ainda se pintando, viraram a cabeça para olhar em
direção das cobras bonitas chegando, e disseram felizes:

--- Acabei de me pintar desse jeito! 

Apesar de não terem dentes como os dos Yanomami, depois de se
transformarem em cobras, depois da metamorfose, os dentes saíram. No
início, não havia cobra, aquelas que picam não andavam no chão, não
havia cobra"-surra, nem coral, nem cobra \emph{maraxa}, nem
cobra \emph{huwëmoxi}. Não havia nenhuma dessas cobras. Lá, onde os
bonitos estavam se transformando em cobras, houve um barulho tão grande
como o de um bando de queixadas, pois as cobras estavam surgindo. As
jararacas, as surucucus, as cobras papagaios e as
cobras \emph{waro} tambémsurgiram. Invadiram toda a floresta. Assim foi. 

Aqueles que haviam convidado as Cobras, os Jalouaca, por causa dos quais
aconteceu a transformação, subiram também ao céu no lugar da
transformação. Os bonitos estavam suspensos. Torurururu! E
trovejou. --- Prohu! --- Chegaram lá. Não estão aqui, nessa terra,
pois andam lá. Queriam viver saudáveis, então estão lá, saudáveis. Não
ficam em baixo. Ficaram em cima. 

Quando as Cobras subiram, o que aconteceu com os amigos delas, os
Jalouacas? Transformaram"-se também em cobra. 

Então, os líderes do primeiro grupo, que se transformaram também
em cobras, ficaram na terra.