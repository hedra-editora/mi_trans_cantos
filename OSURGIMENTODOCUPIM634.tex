\chapterspecial{O {surgimento} {do} {cupim}}{}{}


\letra{Aí}{há cupim}!--- nós dizemos. 
Por causa do rio no qualos Yanomami se afogaram, outros subiram nas
árvores por medo, ensinando"-nos. Ensinaram"-nos a subir. Subiram,
subiram. Conforme o rio subia, eles também subiram, sempre mais, até a
copa das árvores e, em seguida, se transformaram em cupinzeiros, mesmo
não existindo cupinzeiros nesse tempo. 

Os cupins grudaram aos troncos. Os Yanomami se tornaram cupim. A casa
dos cupins fica sentada em cima dos galhos. As casas sentadas são a
imagem daquela que existiu, oriunda dos Yanomami. 

Eles se sentam na forquilha das árvores. Os Yanomami ficaram sentados
nas forquilhas das árvores por medo. Apesar de quererem fugir, eles não
conseguiram. Depois da transformação dos cupins por causa do dilúvio,
aqueles que se afogaram boiaram à deriva na água e se tornaram
jacaré"-açu. Alguns se transformaram em jacaré, outros em peixe, outros
em capivara. Caíram na água, os {Yanomami}. Transformaram"-se
assim pela água. Não foi obra de ninguém! 

Quem os teria feito? Os cupinzeiros eram {Yanomami}. Desde a
transformação, feito isso o cupim está sentado e grudado às árvores nas
beiras de rio.

Uma casa gruda, outra está pendurada, outra está enfiada lá em cima. As
casas de cupim ficaram na posição na qual os {Yanomami} estavam.
É assim. 

São do tamanho de uma criança; uns quase ficaram na terra seca, uns
caíram na água por medo. Os que eram um pouco maiores caíram na água por
medo e se transformaram em cabas\footnote{ Vespas ou marimbondos.}  \emph{xaxa}. Vão se
chamar assim de imediato. Assim se transformaram. Após a transformação,
sua imagem se alastrou. Ocuparam todas as regiões onde moravam os
Yanomami. 

 

 

 
