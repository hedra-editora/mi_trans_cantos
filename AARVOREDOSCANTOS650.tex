\chapterspecial{A {árvore} {dos} {cantos}}{}{}


\letra{N}{ós vamos cantar.} No início, não havia canto, não havia, ninguém
cantava. Onde se erguia a árvore dos cantos, os dois foram caçar. Dois
moços wakusitari -- dois não, um só moço, que a descobriu em sua região.

Os Katarowëteri eram os amigos dos Yãrusi, cujo líder se chamava Yãrusi.
Do outro lado da planície, eles, os Wakusitari encontraram a árvore dos
cantos. 

Outros dizem que foram os Koteahiteri que descobriram a árvore cantando,
e que chamaram os Katarowëteri para pegar os cantos. 

Graças à árvore, os Koteahiteri se enfeitaram com penas de cauda de
papagaio, pintaram"-se com elegância, colocando crista de mutum, e
dançaram. Era uma região bonita e plana onde crescia somente a planta
ária. Eles ocupavam uma bela região. 

Por isso, dois moços koteahiteri foram caçar. 

--- Vamos entrar na mata, lá adiante! 

O irmão mais velho e o irmão mais novo foram caçar. A floresta parecia
mais baixa por causa da luz forte, como a luz do dia na roça. Foram
embora naquela direção, andando. Andavam no meio do brejo, andavam no
meio, ouviram os ecos dos cantos. 

Não havia sujeira no chão onde encontraram a árvore dos cantos dançando,
para frente e para trás. Havia somente areia bonita e muito brilhante. A
árvore dançava. 

--- Ãë, ãë, ãë, e, e, e, e, e, ãë, ãë, ãë, ãë! --- encontraram a árvore
cantando assim. 

--- Ë, aëë, ëaëë, ëaëë, ëaëë, ëaëë, ëaëë! --- cantava a árvore. 

Enquanto isso, o irmão katarowëteri, o filho mais velho, disse: 

--- Õo, irmão menor! Dá pra ouvir um canto, lá onde há uma luz grande
acima do pântano, o som do canto vibra lá, escute isso! Provavelmente é
o som de um grande monstro! Esse som, naquela direção, mais adiante!
Vamos nos aproximar por ali, abrir um caminho no areal! Venha aqui!
Vamos, irmão menor! Vamos logo olhar de perto! 

--- Será voz de gente? --- disseram os dois. 

Onde a árvore dançava, a luz forte batia na areia bonita. 

--- Õoooãaaa! Vamos, irmão menor, vamos! A árvore dos cantos está
dançando, vamos, vamos, vamos até nosso pai, para avisá"-lo! --- disse. 

O irmão menor subiu em uma árvore bonita \emph{matomɨ} inclinada, para
ver se havia gente por perto, se via algum movimento, subiu e ficou
no alto. 

Ali, na areia, a luz brilhava de todas as cores, repousava bem no
centro, e a árvore dançava devagar para frente e para trás, cantando. A boca da árvore era bem bonita, e a árvore dançava para frente e para
trás. 

O irmão menor desceu e disse:

--- Õooãaaa! Irmão mais velho! Irmão mais velho! Nossa! Está lá cantando
e dançando, de uma maneira tão bonita, é a árvore dos cantos! Querido,
parece que essa árvore canta, essa árvore tem cantos bonitos! 

--- Vamos! Vamos até nosso pai! 

Os dois disseram e correram imediatamente. Chegaram correndo.

--- Prohu! Chegamos! 

Eles encontraram esse som e se enfeitaram por causa da árvore dos cantos. 

--- Meus queridos! Enfeitem"-se para pegarem cantos bonitos! --- disse o
líder dos Koteahiteri. 

O irmão mais velho fez o \emph{himou} com o pai, contando"-lhe sobre a
árvore dos cantos.\footnote{  O \emph{himou} é uma modalidade de diálogo cerimonial usada para trazer notícias, ou fazer um convite para uma festa.}

--- Tãrai! Ha! Meu pai! Pai! Olhe! Sou teu filho, olhe! Você não sabe
por que voltei logo correndo! Você nem sabe! Pai! Pai! Pai! Você nem
imagina o canto bonito que meus ouvidos ouviram! De arregalar os olhos!
Meu pai! Meu pai! Meu pai! Você que mora aqui, eu sou seu filho, eu não
lhe diria para proibir as mulheres se enfeitarem! --- disse. 

--- É claro! É claro! Queria ouvir isso mesmo, meu filho mais velho,
querido! --- respondeu seu pai. 

Fez o \emph{himou}: 

--- Vamos! Õoooãaaaaõoãaõoãa! Ele viu uma bonita árvore dos cantos!
Õõoo! --- gritaram. 

Ficaram animados.

 

