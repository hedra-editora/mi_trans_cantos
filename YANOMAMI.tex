\chapter[Para ler as palavras yanomami]{Para ler as palavras\break yanomami}

Foi adotada neste livro a ortografia elaborada pelo linguista Henri Ramirez, que é a mais utilizada no Brasil e, em particular, nos programas de alfabetização de comunidades yanomami. Para ter ideia dos sons, indicamos abaixo.

\bigskip

\begingroup%\footnotesize
\begin{tabular}{rl}
/ɨ/ & vogal alta, emitida do céu da boca, próximo a \textit{i} e \textit{u}\\
/ë/ & vogal entre o \textit{e} e o \textit{o} do português\\
/w/ & \textit{u} curto, como em \textit{língua}\\
/y/ & \textit{i} curto, como em \textit{Mário}\\
/e/ & vogal \textit{e}, como em português\\
/o/ & \textit{o}, como em português\\
/u/ & \textit{u}, como em português\\
/i/ & \textit{i}, como em português\\
/a/ & \textit{a}, como em português\\
/p/ & como \textit{p} ou \textit{b} em português\\
/t/ & como \textit{t} ou \textit{d} em português\\
/k/ & como \textit{c} de \textit{casa}\\
/h/ & como o \textit{rr} em \textit{carro}, aspirado e suave\\
/x/ & como \textit{x} em \textit{xaxim}\\
/s/ & como \textit{s} em \textit{sapo}\\
/m/ & como \textit{m} em \textit{mamãe}\\
/n/ & como \textit{n} em \textit{nada}\\
/r/ & como \textit{r} em \textit{puro}\\
\end{tabular}
\endgroup