\chapterspecial{{Minhocão}}{}{}
 

\letra{A}{ história das minhocas.} Quando a floresta existia, mesmo que a terra
existia: 
--- Vou cavar minhocas! --- ninguém dizia isso. 
Não existia minhoca e, como as minhocas não saíam, ninguém saía, ninguém
pescava depois de tirar minhocas. Era assim. Nós não as deixaremos cair
na água, quando estamos com fome, nós cavamos onde há minhocas, nós as
tiramos, muitas surgirão; para que nós fizéssemos assim, ele morou com a
menina. Lá onde surgiu aquela mulher, a filha de Pokoraritawë ensinará
as Yanomami a não gostar do marido;  às vezes elas não gostam dos maridos. Ensinando"-nos, a filha de Pokoraritawë se zangava demais, pois
estava com medo, não queria seu marido. Apesar de ele ser muito bonito,
a mulher não o queria, a filha de Pokoraritawë fez as minhocas surgirem.
A mulher chegou lá com os dois Minhocões, que comiam o esperma deles
mesmos. Aquele que ela desposou, apesar de ser bonito, foi embora caçar,
até que afinal a mãe falou com a filha: 

--- Filhinha querida, teu marido foi de novo! Vai atrás dele! Vai! ---
ela disse. 

Ela foi bem devagarzinho atrás dele. 

Ele foi, soprou veneno em cuxiús, matou; ele era muito bom caçador,
Paricá. Ela não gostava dele, de Paricá, era o nome do genro de
Pokoraritawë. Minhocão fez os filhotes se multiplicarem com a esposa de
Paricá. Quando seu marido passou, os dois chegaram aonde Paricá estava.
Ele estava longe, adiante, quando a mulher passou perto dos dois
Minhocões, o mais velho e o mais novo. 

Os dois moravam na terra plana e viviam na condição de Yanomami, pois
não existiam minhocas à época. Os pais das minhocas moravam lá, no
início. Eles farão os filhos se multiplicarem. Passando nesse caminho, lá
em baixo, bem longe, Paricá matava cuxiús. As frutas de Minhocão estavam
grudadas. Naquele caminho, as frutas eram numerosas, para atrair a
mulher. Toso, toso, toso, toso! Faziam os restos. Hõti, hõti, hõti!
Faziam assim também. 

Os dois eram muito bonitos, os pais das minhocas: tinham a testa
enfeitada de rabo de cuxiú, guardando a testa, o rosto dos dois era
enfeitado e bonito. Assim era o rosto dos dois. Os dois Minhocões tinham
barba bonita, para parecer o rosto de Paricá e enganar a mulher. Ela
olhou: 

--- Krai! Rae! --- disseram assim. 

Os dois eram esbranquiçados: 

--- Hɨ̃ɨ! Olhe! Olhe! É você? --- disse a mulher bem bonita, com seios
bonitos.

--- Ô! De quem é essa voz?

Como tinha uma clareira, a mulher ficou em pé no limpo. 

--- Não pergunte quem sou! Sou eu! Você! É você mesmo! --- disse a
mulher.

--- Não, não sou aquele que você pensa, eu sou outro! 

--- É você, é seu rosto mesmo, assim que é o seu rosto! 

Ele pronunciou seu nome: 

--- Eu sou mesmo o Minhocão!

--- Não, você não é outro, é você! 

Enquanto ela insistiu em dizer isso, os dois Minhocões logo contaram a ela quem eram. 

Um deles olhou e disse: 

--- Se você diz assim, tire essa folha nova de arumã, aí, aquela folha
enrolada, você a arranca e a desenrola, e você senta em cima, sente"-se
em cima. Coloca sua bunda em cima --- disseram os dois, de um jeito
cantado. 

Rindo, ela correu para arrancar a folha. Pensando que era Paricá, pois
tinha o mesmo rosto, quando ele disse isso, ela arrancou a folha. Depois
de arrancá"-la e desenrolá"-la em um lugar bonito da clareira, onde não
havia nada, ela se sentou em cima, onde estava limpo. Os dois desceram,
os dois desceram rapidamente e copularam com ela uma vez, não várias
vezes, somente uma vez. Apesar de copular com ela somente uma vez cada um, os dois copulavam enquanto o marido estava matando todos os cuxiús,
pois era muito bom caçador, acumulando as presas. 

Ela não o alcançou, andava devagar. 

Depois de ter copulado, não foi nos dias seguintes, mas no mesmo dia,
apareceu a barriga que, apesar de uma vez só, já estava crescendo. 

--- Vai! Vai logo! --- disseram os dois Minhocões, que voltaram para a
morada deles. 

O ventre daquela que estava andando sozinha crescia e crescia. 

--- Vai lá, onde teu marido está matando os cuxiús, ouça os gritos! ---
disse o Minhocão. 

--- Hõhaaa! --- ela ficou pensando.

Depois de falar isso, ela foi bem devagar à direção onde estava seu
marido. Indo lá, o ventre sempre crescia, porque não havia só um filho.
Apesar de serem pequenos, eles estavam acabando com a carne dela. Ela
ficou em pé, enquanto Paricá estava amarrando os cuxiús, ela ficou em pé
lá longe. 

Ele estava voltando. Ele havia matado todos os cuxiús e estava voltando,
depois de carregá"-los, ele estava voltando. Quando voltava, ele viu o
ventre dela enorme de gravidez. 

--- Nunca mexi nessa mulher, e tem filho nesse ventre! --- ele pensou. 

Ele simplesmente pensou. Ele nunca tinha copulado com ela, pois ela não
gostava dele. Ele passou, voltando. Ela voltou sozinha. Ela estava
voltando rindo. Ela estava voltando atrás, sua barriga cresceu
rapidamente. Ela voltava com esse ventre enorme. 

Depois de um dia, o ventre dela estava gigantesco. Ele olhou atrás e viu
a mulher com a barriga enorme. 

--- Hõãaa! É barriga com criança --- ele pensou, e continuou andando. 

--- Hɨ̃ɨɨ! Será que eu já a sujei?

Xiri! Anoiteceu muito rápido. A noite caiu depressa. O ventre estava
cheio. Olha só o suporte dos bichos. Não havia só um! O ventre estava se
mexendo. 

--- Õa, õa, õa, õa! --- diziam, lá dentro. 

A mulher sofria, sofria passando mal, sofria por causa do que acontecia
dentro dela. Doía muito o ventre dela. O marido dela estava deitado na
sua rede, sem olhar para ela, enquanto o ventre dela doía, pois doía
muito, acariciando sua barba e, enquanto a noite logo ficou densa e
grossa, as minhocas saíram.

Weo! Weo! A placenta se derramava como se fosse água, e saíam filhotes de
Minhocão:

--- Ũa! Ũa! Ũa! --- já faziam assim. 

Como parecia voz de criança, ele olhou para as crianças no chão, apesar
de estar deitado na rede, ele olhou. Não havia criança. Ele olhou de
soslaio. Não dava para ver. Embaixo dele:

--- Ũa, ũa, ũa, ũa! --- faziam sem parar. 

Eles nasciam, nasciam, nasciam, nasciam, nasciam, nasciam, nasciam. 

Hɨ̃ɨɨ! Havia tantos montes de minhoca que o fundo da casa sumiu, a vagina
dela estava cheia de minhocas. Depois do trabalho de parto, ela
olhou; ela fez assim. Eles choravam como crianças, chorando de sede
já, eles demonstravam sede: 

--- Sede! Sede!--- diziam, com uma voz de criança. --- Estou com sede! ---
diziam rapidamente. 

--- A criança cresceu tão rápido! --- ela pensou assim. 

Como estavam sempre com sede, ela deu o seio. 

--- Tusu! Suku! Tusu! Suku! --- faziam assim enquanto mamavam. Ela
fez assim. Como as minhocas faziam isso, ele ficou esperto. Ele
entendeu: 

--- Hɨ̃ɨ! --- ele pensou. 

A mãe dela chegou correndo. Apesar de olhar, ela não as viu
imediatamente. Apesar de escutar o choro de criança, ela olhou e voltou
a deitar. 

Deitada, a mãe das minhocas as cobriu, cobriu, cobriu, cobriu, cobriu.
Amanheceu. Como a filha estava indo de manhã cedo, ela falou para sua
mãe, enquanto o marido estava ali pensativo.

--- Mãe! Não descubra o que eu cobri no fundo da casa. Não fique olhando
o fundo da minha casa! 

Havia tantas minhocas! Elas se embolavam, zoando, porque estava cheio. 

--- Não olhe o fundo da minha casa. Não descubra o que eu cobri! --- ela
disse, e saiu. 

Xiriririri! E sumiu. Enquanto isso, a mãe levantou da rede. 

--- Por quê? Onde está essa criança, que deveria estar no colo,
recém"-nascida? Vai chorar muito, assim! --- ela pensou, e correu até a
casa. 

Ela foi logo. Ela correu e descobriu o que estava onde a filha morava,
aquelas minhocas, todas mexiam a cabeça ao mesmo tempo. 

--- Xiririririri! Sede! Sede! Sede! Avó! Sede! --- eles a chamavam de
avó. --- Avó! Sede! Avó! Sede! Avó! Sede! --- todos diziam. 

--- Hɨ̃ãaaaaë! --- ela gritou logo. --- Hɨ̃ãaaaë! Só você para fazer surgir
aquilo! Por isso! Você não trata bem seu marido! É por causa desses
bichos estranhos que você não conseguiu dormir! --- ela disse. --- Vai! Meu
genro! Enquanto eles se mexem assim, derruba logo essa lenha, faz um fogo
grande para ela! --- disse a mãe. 

Ela mandou queimar a filha viva! Depois de ela dizer isso, ele desceu da
rede. Ele não demorou: derrubou aquele carapanã-uba.

Kraxi! Kraxi! Kraxi! Krao! Torou! Fazia lenha para cremá"-la. Enquanto
fazia lenha, ela voltou. Ela tinha ido tomar banho sem perceber, ela passou
onde ele estava partindo a lenha. Ele virou as costas, onde ele estava fazendo
lenha. Ele nem olhou. Ela se deitou, encolhida. 

Pou! Pou! Pou! Ele amontoou muita lenha. Pou! Pou! Pou! Ele pegou
brasas para acender o montão de lenha, ele fez aumentar o fogo. Como a
lenha era seca, o fogo pegou logo. 

Weee! Ele fez uma cerca, fez para ela. Depois, ele correu atrás dela.
Ela nem se levantou, ele gritou para pegá"-la, pois queria a cremar viva. 

Weeeee! Ela estava deitada bem reta. Ela nem reagiu, ele correu a carregando
em direção do fogo, e ela chorava: 

--- Ëaë! Ëaë! Mãe! Pai! 

As pernas dela estavam balançando, dando impulso. Ele a jogou no meio
do fogo. 

Pou! Ele pegou outra lenha que estava no chão e amassou, amassou com
força. 

Ëëëaaaëëë! Proto! O fogo queimou, enquanto cremava, a sogra dele correu
em cima dos minhocões para queimar os feios. Ela correu para pegá"-los.
Ela já tinha colocado água em cima do fogo em uma panela de barro para
cozinhá"-los. Ela correu com uma vasilha de água quente em direção das
minhocas cobertas. Ela jogou a palha de coruá que as cobria: 

Weeeo! Os minhocões gritavam:

--- Õiii, õiii, õiii! 

--- Avó! Couro encolhido! Couro encolhido! Couro encolhido! Couro
encolhido! --- diziam atordoados, chamando"-se de pele encolhida. 

--- Avó! Couro encolhido! Avó! Couro encolhido! Avó! Couro encolhido! ---
diziam os pedaços, arrebentados. 

Olha só os montões de pedaços! Os pedaços estavam correndo logo, e
ocuparam toda a floresta, os minhocões. Ficaram ocupando a floresta, os
arrebentados, correndo logo pra todas as beiras de rio, entraram depressa
no fundo da terra. 

Depois de acontecer isso: --- São minhocões! --- Dizemos. Foi assim que aconteceu.
Não existiam minhocas. Foi com ela que se multiplicaram. Nós as faremos
cair na água para nós comermos peixes. A minhoca não apareceu do nada. 

Foi depois de os dois Minhocões copularem com ela e multiplicarem seus
filhotes, que foram embora com os pais. Os filhotes não moraram onde
foram cremados, nem ficaram ali perto. Os dois foram logo. Assim foi. Desde
que aconteceu, quando cai a chuva:

--- Tëɨ, tëɨ, tëɨ, tëɨ tëɨ! --- dizem seus pais, de onde estão. 

Assim foi a história.

