\chapterspecial{O {corte} {dos} {cabelos}}{}{}
 

\letra{Q}{uando} não havia \emph{napë}, sofriam de ter o rosto fechado pelos
cabelos que desciam, tinham o rosto como o de mulher por causa dos
cabelos. Ele fez o bambu \emph{sunama} e o
bambu \emph{waharokoma} aparecerem\emph{.} Os Yanomami cortavam os
cabelos com ponta de tacuará. Quando não o encontravam, usavam o
bambu \emph{uhe}. Rasgavam"-no e cortavam os cabelos com isso, faziam o
corte com esses pedaços. Eles se davam esses pedaços de má qualidade,
pois não havia \emph{napë}. As mulheres sofriam com o sangue do corte,
quando faziam assim, cortavam a testa, como faziam assim, eles sofriam.
No início não havia tesoura. 

Qual é o \emph{napë} que apareceria e inventaria aquela tesoura?

No início, se cortavam mutuamente o cabelo com pedaços de tacuará
afiados. Partiam o bambu \emph{sunama}, com o qual se cortavam o cabelo
mutuamente, com o fio da lâmina. Kreti! Kreti! Kreti! Cortavam"-se o
cabelo mutuamente. Assim que faziam entre eles. Também não havia facão. 

Cortavam também a carne com pedaços de tacuará \emph{sunama}, no início.

 
