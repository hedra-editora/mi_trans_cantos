\chapterspecial{O {pássaro} {siikekeata}}{}{}
 

\letra{K}{eora,} kɨrɨɨɨ, keora kɨrɨ!--- disse assim. 
Os dois fugiram ensinando o ritual do \emph{yaɨmou}. Apesar de serem
somente dois, faziam festa, ensinaram a encher os cestos de fruta
conori. Não sabiam matar anta, porém tinham muita anta moqueada. 

Como se chamavam os dois avós de Siikekeatawë? O neto deles se chamava
Siikekeatawë. 

\emph{Siikekeã, keã, keã}! Vocês escutam esse canto de passarinho? Era o
nome dele. Não era Yoawë. Era outro irmão mais velho. Yoasiwë era o nome
do irmão mais novo. 

As famílias yanomami em geral são numerosas. 

Esses dois apareceram na seqüência desta história. Sim, os dois
apareceram. Aquele que eles chamam Yoahiwë, aquele se
tornou \emph{napë}, os dois foram ao rio Tanape. Omawë e Yoahiwë se
tornarão \emph{napë}. Não foram esses dois que ficaram. Não foi Omawë, o
irmão mais velho e bonito. Nem Yoasiwë. Não foi aquele
passarinho \emph{uxuweimɨ} bonito que ficou.\footnote{ N\,E\,Ver as histórias de Omawë.} 

Trata"-se aqui de outros dois. Ficaram somente os de aparência triste e
velha. Esses dois ficaram.

Onde fica a foz do rio, cuja parte inferior olhamos? Onde se encontram
os dois rios? Foi nesses dois rios que os dois se dividiram. É assim. É essa a história dos dois que se dividiram. 

Esses dois irmãos mais velhos ficaram, ficaram fazendo festa, ensinaram
os descendentes a fazer festa. Terminou a festa, a festa acabou quando
juntaram a comida, colocaram a carne de anta em cima dos conoris.
Enquanto isso, eles cheiraram paricá. Ensinaram o ritual
de \emph{yaɨmou}, apesar de estarem sozinhos. Os dois conversaram,
fazendo o ritual de \emph{yaɨmou} no meio do xapono. 

--- Sou teu irmão e pergunto, como vamos falar? Nós vamos discutir e,
depois, nos fazer comer mesmo!--- disseram os dois, ensinando. 

Disseram o que dizem os Yanomami quando fazem o ritual de \emph{yaɨmou}.
Os dois deram exemplo aos Yanomami. 

--- Vamos encher a barriga até cair!--- falavam brincando, como se
houvesse muita gente ao redor. 

--- \emph{Aë, aë, aë, aë, aë, aë, heeeee}--- um deles diz. 

Enquanto dizia isso, o neto deles saiu levando o arco \emph{haowa}.
Ouviu a voz forte do pai. O menininho escutou a voz forte do seu pai, a
voz vinha subindo da curva do rio. Apesar de ele não ser de grande
tamanho, apesar de ele ser pequeno, a voz forte espantou os dois até os
expulsar. 

Enquanto flechava passarinhos, o neto correu até certa distância e,
enquanto flechava, a voz forte surgiu. 

Os dois avós faziam o ritual de \emph{yaɨmou}, se batendo no
corpo. \emph{Hëë, hëë, hëë, haëëë, haëëë, haëëë}! (Eu mesmo sou surrado
pelos meus parceiros ao fazer o ritual de \emph{himou}!) 

Enquanto faziam assim, o neto rodeava para matar passarinhos. Nesse
momento, a voz grande surgiu. A voz não era fina. 

--- \emph{Siiiikekeã, kea, kea}, rasgar a pele, rasgar, rasgar!--- dizia
a voz, chegando. 

A floresta estava tremendo com essa voz. 

--- \emph{Siiiiikekea, kea, kea}! dizia a voz, chegando. 

Apesar de ser pequeno, ele estava andando ali, ouvindo essa voz. 

Enquanto os dois avôs discutiam, o neto retornou. Na entrada, soltou o
arco. Soltou"-o de medo. 

--- Avô!Avô! Parem com esse barulho! A voz terrível do monstro se
aproxima! Ele vem rasgar a pele de vocês, a voz já está perto, avô,
parem!--- disse para os dois. 

Quando os dois pararam, assustados, pararam de repente, a voz surgiu
naquele instante. Quando ficaram silenciosos, ouviu"-se logo a voz
forte: 

--- \emph{Siiiiikekea, kea, kea}!--- disse aos dois. 

--- \emph{Hɨ̃ɨɨ}!--- gemeram os dois, e se levantaram assustados. 

O que lhes aconteceu?

--- Vamos! Vamos, querido! O monstro vai rasgar nossa pele, parente
querido, vamos, vamos, depressa!--- disseo mais velho, cujo nome já
pronunciei. 

Enquanto a voz dizia isso, o irmão mais velho também queria se
transformar. 

--- Vamos, venha, meu irmãozinho, venha, meu netinho! Vamos depressa! O
neto rodava na frente dos dois. 

--- \emph{Keora kɨrɨ! Keora kɨrɨ! Keora kɨrɨ}, vamos cair na água, lá
embaixo!--- disseram logo os dois. 

Os dois disseram isso, embarcaram e voaram acima da água. Lá, os dois
caíram rio abaixo. Enquanto os dois prosseguiam, a cauda vermelha do
neto saiu. \emph{Prohu}! 

O neto seguiu os avós, aqueles que voam. Por que ele tem esse nome de
Yoasiwë? Quando os passarinhos pousam em galhos fincados na água, a
cauda deles não é vermelha? O neto se transformou nesse passarinho. Onde
ele se transformou, onde foram os dois avós, a imagem do neto ficou, se
multiplicou e ficou voando acima das águas. Os dois avós caíram, levando
à frente deles seu verdadeiro neto, de quem restou somente a imagem.
Assim foi. 

Onde os dois avós caíram e onde a sua imagem ficou, se soterra o fim do
rio. É lá que os dois moram, eles não morreram e ainda moram lá. Eles
não morrem de doença. 

Os dois ensinaram o ritual de \emph{yaɨmou}, da mesma forma que faziam;
é por isso que nós, Yanomami, fazemos festas. Nós perpetuamos os
rituais. Não foram outros que nos ensinaram a fazer festa. 

 
