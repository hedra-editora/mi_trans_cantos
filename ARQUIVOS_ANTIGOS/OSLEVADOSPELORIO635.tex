\chapterspecial{{Os} {levados} {pelo} {rio}}{}{}
 

\letra{H}{avia um pajé} chamado Xiritowë, queveio a se chamar Keopëteri. Os
descendentes moram lá ainda. Os Xiritowëteri, depois, se chamaram
Keopëteri. 

Qual era o nome do rio onde eles afundaram, aquele que deu o nome de
Keopëteri? Esse rio se chama Xitipapɨwei. Eles bebiam dessa água, que
cobriu tudo. Caíram nas águas do Xitipapɨwei. Eles não morreram. Ainda
existem ali. 

Um dia, Xiritowë mandou seu genro buscar os visitantes
numxapono\footnote{  É o nome da casa coletiva circular onde vivem os Yanomami. Cada casa
dessas corresponde a uma comunidade ou assentamento. [\textsc{obs}. Não estou
grafando em itálico porque considero que pode ser usada tranquilamente
como palavra portuguesa].}  amigo para dançar durante a festa, e assim
nos ensinou a nos chamarmos e a nos convidarmos mutuamente. 

O genro de Xiritowë foi convidar os parentes e amigos deles, os
Anahupɨweiteri para dançar durante a festa, assim como fazemos até hoje.
Ele correu até o xapono dos Anahupɨweiteri. 

O pajé Xiritowë chamou seus conterrâneos para a festa sem imaginar que o
rio inundaria o xapono; ele pensava que todos iriam morar lá para
sempre. 

Como se chamava esse rio {antigo}? Esse rio se chamava
Xitipapɨwei.Aqueles que as águas cobriram, antes bebiam dessa água. 

 Xiritowë tornou a se chamar Keopëteri (os afundados). Depois de eles se
afundarem nas águas desse rio, tornaram a se chamar Keopëteri, porque
caíram nas águas desse rio que os levou. Assim, ainda moram lá. Esse rio
se chama Xitipa; os Keopëteri caíram nas águas do Xitipapɨwei. Eles não
morreram. Existem ainda. 

A filha de Keopëteri e seus parentes não se afogaram, apesar de estarem
no fundo do rio, o rio os levou. Eles vivem sempre lá. Tornaram"-se
eternos. Tornaram"-se esses monstros que nunca morrem; eles afundaram.

 

