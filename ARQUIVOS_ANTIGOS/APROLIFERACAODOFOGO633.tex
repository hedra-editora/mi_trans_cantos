\chapterspecial{A {proliferação} {do} {fogo}}{}{}
 

\letra{E}{u vou contar} a história de Jacaré. Seus conterrâneos sofriam por causa
da escuridão à noite, porque não conheciam o fogo e, então, comiam cru.
Depois de apanharem frutas \emph{kaxa}, eles as comiam cruas, pois não
havia fogo. A região chamada Kaxana era a região de Jacaré. Ele ocupava
essa região. Ele bebia a água do rio Kaxana. Era o nome dessa região.
Ele comia as frutas \emph{kaxa} cozidas às escondidas, aquele que
detinha o fogo na sua boca. Essa região Kaxana fica no centro da
floresta. No meio dessa região, há o rio Kaxana. 

Os jovens que moravam com Jacaré estavam sofrendo e delirando por causa
da comida crua. Ele comia cozido sozinho. Ele não oferecia a eles. Até o
paladar das mulheres sofria com a comida crua. A história dos
companheiros de Jacaré, que, depois de refletirem, encontraram o fogo,
guardado por Jacaré, ocorre no meio da nossa história. 

Alguém acordado à noite ouviu o som baixinho daquele que mastigava
escondido comida frita. \emph{Kãrɨ, kãrɨ, kãrɨ}! Por causa de sua boca
emitindo esse som baixinho, um deles percebeu o que estava acontecendo. 

--- Será que ele está comendo algo frito?

Ele pensou assim, apesar de não ver lenha queimada no chão, como
acontece quando a gente acorda e, ainda deitado, olha para o chão. 

Depois de torrar os alimentos na sua boca, ele os comia com sua esposa à
noite. \emph{Kãrɨ, kãrɨ, kãrɨ}! Os dentes dos dois faziam esse som
baixinho. 

Depois de seus companheiros acordarem, falavam de Jacaré, baixinho. 

--- \emph{Hoaaaaaaaa}! Vamos, vamos procurar! --- disseram logo. 

Não foram nossos antepassados que descobriram o fogo. Nós não conhecemos
o fogo por ele ter aparecido de repente. Nossos antepassados sofriam,
eles endoideciam por causa da comida e pareciam doentes. Sofriam e
endoideciam por comer carne crua. 

Salvaram"-se com o fogo de Jacaré, que pegaram e espalharam. 

Eles viviam sempre tristes; o fogo quase não saía, quase não existia.
Viveriam sempre tristes se o fogo não tivesse existido e sido
repartido. 

Jacaré ensinou os \emph{napë} a fazer fogo; eles o usaram porque Jacaré
lhes ensinou, pois ele guardava o fogo na sua boca.

Depois de acender as brasas em um tipo de forninho, ele colocava a
comida nas brasas. A comida cozinhava escondida em cima das brasas; ele
fritava as caças e as frutas às escondidas, assim ele fazia. Ele
guardava o fogo com ciúme, quase não revelava o fogo. 

Todos se juntaram para descobri"-lo. Agruparam"-se, onde Jacaré comia
queimado e guardava as folhas dos embrulhos. Ele as enterrava, cobria"-as
com terra; cavava o chão levemente e colocava as folhas queimadas; assim
fazia. Sozinho, saboreava a comida cozida. É um jacaré, como a gente
diz.

Mas ele cedeu o fogo? Não, ele sovinava o fogo. Vocês só veem um pedaço
da língua dele por causa do fogo. É um jacaré. A língua não queimou à
toa, o fogo acabou com a língua, ficou somente um pedaço no fundo, pois
ele guardava o fogo na boca. 

Avisaram todos os xapono, convidando"-se a se reunirem. Chegaram e se
reuniram. Jacaré não ficou sabendo e saiu à procura da
fruta \emph{kaxa}, enquanto eles se reuniam. 

Onde se revelará o fogo, queriam se alegrar rapidamente com a comida
cozida; conseguiram encontrar lascas queimadas de comida. Revelou"-se o
fogo com Jacaré; os \emph{napë} e os Yanomami terão o usufruto do fogo,
por isso comemos cozido.

No início, quase não conhecíamos a comida cozida. Onde nossos ancestrais
comeram cozido? Onde comiam cru com sangue? Quase todos nós comíamos
cru! Você escuta as minhocas e os minhocões, eles estão
aliviados, \emph{Tɨ̃ɨɨ}! \emph{Tɨ̃ɨɨ}! Quase você comia esse tipo de
minhocões, quase você sofreria, assim como os antepassados sofriam. 

Reuniram"-se, como nós agora. Na reunião, proibiram Jacaré de sair. 

--- Vamos, avô! --- seus netos disseram. 

Um de seus netos, aquele tipo de calango que fala
assim, \emph{Serororo}!, aqueles calangos pequenos que caem na água e
que fazem assim, \emph{Serororo}! Nós chamamos de \emph{temoa}. Era um
dos netos dele, quando era Yanomami. 

--- Vamos, meu tio! Fique parado! Você vai escutar. Escute os que se
reúnem. Deixe para ir amanhã de novo à procura de comida! --- disse seu
neto. 

--- \emph{Haaa}! --- disse ele com uma voz rouca. 

--- \emph{Hai! Hai!} Por quê? --- ele não perguntou. 

--- \emph{Ho}! Está bem! --- disse. Assim disse logo. 

--- \emph{Hëëë}! --- disse ele, da mesma forma que diz hoje. 

Enquanto se dirigia a ele, ele ficou parado. 

--- \emph{Ho}! Sim! Vamos esperar logo! --- disse à sua esposa. 

Ela tem uma voz harmoniosa, como a que se escuta nas cabeceiras dos
igarapés. \emph{Pẽi! Pẽi! Pẽi!} Os que falam assim, na beira d'água, os
que sempre ficam dentro da água, os \emph{pëipëimɨ}. Era a esposa de
Jacaré. Apesar de ser esposa de Jacaré, era parecida com a perereca que
tem desenhos na coxa. 

Quando o neto disse aquilo, eles pararam, ele e sua esposa. \emph{Hɨ̃ɨɨ}!
Ele sabia de quê se tratava. 

--- Eu não revelarei o fogo que eu possuo! --- pensou logo. 

Ele se zangou logo, sem razão. Ele foi dormir, para não dizer nada
àqueles que estavam reunidos. \emph{Hɨ̃ɨɨ}! Ele afastou sua rede, virou
seu rosto para o outro lado e adormeceu. Não se mexia. 

Os outros queriam fazer rir Jacaré e se revezavam. Os \emph{kiritari} se
revezaram. \emph{Ão, ão, ão, ão, ão, õoooo!} Eles tentaram, mas ele não
se mexeu. Aquele que possuía o fogo, cujas brasas brilhavam, guardou a
boca bem fechada e parecia não ter boca. 

Cada pássaro ia no meio do xapono dançar, se revezando.
Os \emph{huimiri}, os \emph{taɨsarakari} e os galos da serra iam deitar
com ele, se revezavam. Os bicos"-de"-brasa se sentavam no chão, sentavam,
sentavam, mexendo as asas; queriam fazê"-lo rir e se revezavam. A esposa,
de costas, disse logo: 

--- Não se mexa! Fique aí quieto! --- disse ela.

Em seguida:

--- Não se mexa! Durma! --- Falava ela, do seu canto.

--- Vocês vão fazer assim! --- Jacaré não disse, apesar do barulho. 

Bem depois, eles cansaram. Ele não se mexia. Ficaram cansados, os
coitados se mexeram tanto, tiveram tantas dores, que começaram a
sofrer. 

Beija"-Flor não se levantou logo. Ele estava no centro, afastado com
Tohomamoriwë, irmão dele. Eram os dois: o mais velho, Beija"-Flor, e o
mais novo, Tohomamoriwë, daqueles beija"-flores pequeninhos. Beija"-Flor
agiu primeiro. 

--- Irmão maior, sua vez! Você tenta logo, eu vou olhar primeiro! ---
falou o irmão mais novo. 

Apesar de o irmão mais velho se mexer, Jacaré não reagiu.

--- \emph{Sĩo! Sĩo! Sĩo! Tõu, tõu, tõu!} --- disse.

Mexendo"-se à frente de Jacaré, ficando parado, ele colocou peninhas
brancas no seu ânus, nas suas patas, as peninhas estavam repartidas
igualmente nos dois lados do ânus. A língua de Beija"-Flor saía e os
olhos de Jacaré se abriram. Deu uma olhada e adormeceu de novo. 

--- Não com você! --- pensou. 

Beija"-Flor mexeu duas vezes e quando todos terminaram, ficaram cansados.
O irmão mais velho disse ao mais novo: 

--- Vai, irmão menor, tua vez, depressa! Com você é bem capaz de ele
rir.

Aí o irmão mais novo disse:

--- \emph{Waooo!} Olhem isso! Sou eu mesmo! Olhem para mim! --- disse
ele. 

Todo o mundo olhou. Bico"-de"-Brasa estava triste, sentado, como um
doente, aguardando o fogo para pegá"-lo logo. Aquele que estava esperando
se levantou e aproximou"-se. 

--- Bem, olhem só como eu faço: \emph{Sĩo! Sĩo!} --- dizia a certa
distância, de onde se levantou. 

Parecia o som do carapanã \emph{tëërëkë.}

--- \emph{Sĩo! Sĩo! Sĩo! Tëɨ! Tëɨ! Tëɨ!}

Um som bonito começava a ser ouvido e o dono do fogo arregalou os olhos,
olhou para o que acontecia.

Beija"-Flor logo se levantou, para dançar à frente de Jacaré.

--- \emph{Wão!} Você vai ver!

Ele ficou à frente de Jacaré, continuando a dizer:

--- \emph{Sĩo! Sĩo!}

Parecia voar para sempre. 

--- \emph{Tëɨ, Tëɨ, Tëɨ!} --- ele falava. 

Ele parecia pendurado com as pernas abertas. Ele fazia assim: a cauda
dele ficou virada para cima, e os dois irmãos ficaram um ao lado do
outro. Eles faziam como se fosse um ritual. 

Nessa altura, Jacaré se levantou. Ele estava sério, mas se endireitou e
sentou na rede, e deu uma gargalhada. Com as gracinhas de Tohomamoriwë,
ele entregou o fogo. Tohomamoriwë estava na altura dos olhos de Jacaré: 

--- Ho, ho, ho, ho! --- Jacaré riu. 

\emph{Prohu}! A brasa pulou. \emph{Tou}! Bico"-de Brasa esperava o fogo e
o apanhou. \emph{ɨɨɨ, krihi, krihi}! Ele apanhou o fogo, que queimou seu
bico, por isso o bico dele é vermelho, pela queimadura do
fogo. \emph{Hɨ̃}! Como o fogo era pesado, ele não conseguiu voar alto,
quase caiu de volta com o fogo. 

--- \emph{Xĩapo}! Disse ele.

Ele estava esperando, empoleirado mais em cima. Ele foi levar o fogo.
Wẽooo! Pois ele é maior. Tu tu tu tu tu tu tu! Levou as brasas em cima
da árvore abiorana murcha, ele as colocou na ponta do tronco da árvore. 

A mulher de Jacaré se levantou. Quando Bico"-de Brasa quase apanhava o
fogo, ela jorrou urina. Apesar de a mulher de Jacaré jorrar sua urina,
Japu levou o fogo mais alto e a urina não conseguiu alcançá"-lo. Enquanto
o fogo estava em cima e não apagava, ela não acertou o fogo, a urina não
alcançou. Ela disse:

--- Vocês pegaram o fogo, então vocês chorarão quando cremarem os seus
mortos, vocês sofrerão e chorarão pelos seus mortos cremados! 

Ela falou verdade, pois quando morremos, nos cremamos; ela disse a
verdade: nós praticamos a cremação e nos cremamos. Ela disse que era
para ser nossa tradição. 

--- Eu vou ao igarapé e ficarei feliz lá com meu marido para sempre ---
disse --- Vocês sofrerão com o fogo. Ele se tornará eterno. O fogo
derreterá seus olhos! 

É verdade o que ela disse, a esposa disse a verdade. Nós nos cremamos,
nossa carne queima, ela falou certo. Se isso não houvesse acontecido,
não nos cremaríamos. Dito isso, ela e seu marido, Kruxu! Kopou! Os dois
foram às águas para nada de ruim acontecer, para eles não ficarem
doentes, não pegarem diarréia, nem dor de cabeça, nem conjuntivite, nem
terem dor nas pernas, nem conhecerem a malária. E assim serão. 

Eles tem problemas de dentes como nós? Não, tem não! Não conhecem dor de
dente. 

Esses eventos aconteceram para que seja assim. Eles ainda estão
felizes. 

--- Pĩri! Pĩri! Pĩri! Pĩri! Pĩri! Pĩri! Pĩri! Pĩri! Pĩri! Pĩri! Pĩri!
--- cantou a esposa até chegar às águas. 

 Eles pegaram o fogo e os dois moram ainda nas águas.
