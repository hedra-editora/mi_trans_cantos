\chapterspecial{O {surgimento} {da} {flecha}}{}{}
 

\letra{A}{ história} da flecha. Aconteceu o seguinte. Tinha o dono. Não foi outro
que depois de abrir um tipo de roça plantou as flechas. Onde morava o
dono, parecia um flechal, essas flechas que eles plantaram em seguida em
todos os xaponos\emph{.}

Assim que é, porque ele é o dono mesmo. Aquele que descobriu a flecha se
chamava Xororiakapëwë, é seu flechal, fará atirar as flechas, aquele que
descobriu as flechas, era a imagem das pequenas andorinhas que voam
acima da água. 

Xororiakapëwë descobriu as flechas, fez as flechas \emph{hauya}. Graças
a ele, os Yanomami descobriram a flecha e pegaram"-na. O limite do
flechal fica na boca do rio subindo; é seu flechal, não é de Yanomami.
Eles pegaram as flechas e as espalharam. Ele fez as flechas se
multiplicarem. 

Os Yanomami não tinham flechas, depois de pegarem"-nas e plantarem"-nas,
eles guerrearam. Antes eram desprovidos, não tinham flechas, eles
flechavam com dalas pequenas de arumãs em penas, aquelas flechas nativas,
ou de caule de planta \emph{tomɨ si}. Ofereciam"-se essas flechas de má
qualidade, pegavam haste de caranarana parecidas com flechas, amarravam
penas na extremidade e flechavam com essas flechas de má qualidade. Não
existiam flechas de verdade. Foi por ele que os Yanomami se flecharam,
pois ele as fez. É o dono mesmo. 

 
