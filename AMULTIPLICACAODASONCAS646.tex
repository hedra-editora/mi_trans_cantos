\chapterspecial{A {multiplicação} {das} {onças}}{}{}
 

\letra{E}{m seguida,} segue a história daquele que fez as onças se multiplicarem.
Ele foi à direção certa. Existe nos buracos de pau. Onde havia buraco de
pau, outro tipo de onça existia, a onça \emph{ɨrahena}. 

Não foi obra de ninguém! Eles tinham um xapono como este. Era o mesmo nome
daquele que a tirou do buraco. Aquele que tirou a onça \emph{ɨrahena},
onça parecida com jia, depois de tirá"-la, ele se alegrou com a pele
pintada; depois de ele arrancar as folhas, as onças habitaram toda a
floresta. 

Ele chegou ao xapono. Haviam queimado. Nesse lugar, a roça estava
próxima. Ele plantou as jias no lugar queimado. Ele plantou. Apesar de
ser jia, ela não apodreceu, pois era onça. Onde ele plantou um pouquinho,
ao final do dia, quando a floresta escureceu, da mesma forma que os
capins tem flores, essa flor de onça também desabrochou. 

A onça grande começou a surgir. Onde caíram as sementes, as onças se
levantaram. Os Kaxanawëteri moravam no centro dessa região. São aqueles
que plantaram a onça \emph{ɨrahena}. 

Surgiram as onças suçuaranas, as onças suçuaranas vermelhas, as grandes
onças e as onças pretas. As onças exterminaram os habitantes do xapono
onde haviam plantado as onças, e cujo nome eu dei. Ninguém sobreviveu. 

Elas são famintas de carne, e não foi só uma que andou. Logo comeram os
habitantes. Esses antepassados não tiveram descendentes, pois nenhum
conseguiu fugir. Nenhum sobreviveu. Exterminaram todos. Nenhum. Não foi uma onça só. Em um dia, exterminaram todos. Comeram também aquele que tirou
a onça. Ele morreu também. 

Depois de exterminar todos, a onça continuou a surgir na terra
dos \emph{napë}, apesar de essa terra se estender bem ao sul. Não foi obra
de ninguém. As onças apareceram onde foi plantada a onça. Apesar de ser
jia, a jia não apodreceu. Lá, a onça ficava dentro, a
onça \emph{ɨrahena}. 

O que segue, é a história das onças que comeram muita gente. 

Eles moravam perto da serra Yamaro e se chamavam Yamarowëteri. Chamaram
o xapono deles Yamaro. Apesar de eles não terem plantado urucuzeiros,
havia muitos no meio, por isso se chamavam assim. É outro nome para
urucuzeiro. As onças os comiam também nas regiões vizinhas. 

Os vizinhos um pouco mais distantes eram os Sementes"-de"-Urucu. Eles
bebiam a água do rio Ximono. A voz deles era fina. 

Após o xapono deles, havia outro grupo. Eram os vizinhos. Todos tinham
os cabelos vermelhos. Os cabelos deles era de um vermelho bem forte. Os
vizinhos deles eram os Ɨranawëteri. Chamaram o rio, do qual bebiam a
água, Ɨrana; por isso se chamavam Ɨranawëteri. Assim faziam nossos
ancestrais.
