\textbf{A árvore dos cantos} faz parte do segmento Yanomami da coleção Mundo Indígena --- com \textit{O surgimento dos pássaros}, \textit{O surgimento da noite} e \textit{Os comedores de terra} ---, que reúne quatro cadernos de histórias dos povos Yanomami, contadas pelo grupo Parahiteri. Trata-se da origem do mundo de acordo com os saberes deste povo, explicando como, aos poucos, ele veio a ser como é hoje. A história que dá nome a este volume fala sobre o surgimento do canto, que nasceu a partir de uma árvore. Mas reúne também outras narrativas: sobre o surgimento da cobra, da flecha, a multiplicação das onças. 

\textbf{Anne Ballester} é coordenadora da \textsc{ong} Rios Profundos e tem experiência de vinte anos junto ao povo Yanomami do Rio Marauiá. Trabalhou como professora na área amazônica, e atuou como mediadora e intérprete em diversos \textit{xaponos} do rio Marauiá. Foi coordenadora do Programa de Educação quando ajudou a criar a \textsc{secoya} (1998), então comprometida com a garantia do sistema de saúde indígena. Dedicou-se à difusão da escola diferenciada nos \textit{xaponos} da região, como também à formação de professores Yanomami, em parceria com a \textsc{ccpy}/\,Roraima, incorporada atualmente ao Instituto Sócio Ambiental (\textsc{isa}). Ajudou a organizar cartilhas monolíngues e bilíngues para as escolas Yanomami, a fim de que os professores pudessem trabalhar em sua língua materna.
% nasceu em 1955 na França e viveu por 24 anos com os Yanomami. Enquanto ativista, trabalhou como agente de saúde no combate à malária, além de alfabetizadora em língua yanomami e professora de português para jovens e adultos em posições de liderança indígena. É cofundadora da \textsc{ong} Rios Profundos. 

\textbf{Mundo Indígena} reúne materiais produzidos com pensadores de diferentes povos indígenas e pessoas que pesquisam, trabalham ou lutam pela garantia de seus direitos. Os livros foram feitos para serem utilizados pelas comunidades envolvidas na sua produção, e por isso uma parte significativa das obras é bilíngue. Esperamos divulgar a imensa diversidade linguística dos povos indígenas no Brasil, que compreende mais de 150 línguas pertencentes a mais de trinta famílias linguísticas.


